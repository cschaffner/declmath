\documentclass{omdoc}

\usepackage{modules}
\usepackage{cmath}

\newtheorem{lemma}{Lemma}

\makeatletter
% Provides additional metadata for index entries
% #1: OpenMath name of symbol
% #2: Options:
% - description: Short human readable text describing the symbol (for symbolindex+popups)
% - placeholder: Alternative text code to show in symbol index instead of symbol
%                (required if symbol macro takes arguments)
% - nopage: true iff the symbol is not defined in the paper (no \indexdef call). Otherwise: warning
\newcommand\indexinfo[2]{\typeout{Index info for \mod@id.#1 configured: #2 (but ignored)}}

\begin{module}[id=myPaper]
  \symdef[name=powerset]{power}[1]{2^{#1}}
  \termdef[name=powerset]{powerset}{power set}
  \indexinfo{powerset}{description=Power set of $S$, placeholder=\power S}
  
  \symdef[name=cardinality]{card}[1]{\lvert#1\rvert}
  \termdef[name=cardinality]{cardinality}{cardinality}
  \indexinfo{cardinality}{description=Cardinality of $S$, placeholder=\card S}
\end{module}

% This command marks the place in the paper where #1 is defined (for indices)
\newcommand\indexdef[1]{\textsuperscript{[#1]}}

\begin{document}

\section{Example section}

\usemodule{myPaper}

\begin{definition}[title=\capitalize\cardinality]
  The \emph{\cardinality}\indexdef{cardinality} of a set $S$ is denoted $\card S$ and tells us how big it is.
\end{definition}

\begin{definition}[title=\capitalize\powerset]
  Given a set $S$, the \emph{\powerset}\indexdef{powerset} of $S$ is denoted $\power S$.
\end{definition}

\begin{lemma}[Cantor]
  $\card S\neq\card{\power S}$.
\end{lemma}

TODO: I would like both a symbol index and a normal index based on the markup above.

\end{document}
